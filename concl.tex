% !TEX root = ln_diss.tex
\chapter{Conclusion} \label{chap:CC}

This chapter draws some conclusions from the work presented in this dissertation, and provides a comprehensive outlook.

\section{Summary}

The present work has allowed to build up various finite element solvers for the wave equation that extend the capabilities of widely available, commercial or not, similar solvers.

In the first chapter, the finite element method for the wave equation have been introduced, and its implementation in the FES-3D package has been validated by an extensive comparison with a commercial package. Several boundary conditions have been tested in order to allow the analysis of many electromagnetic problems, for instance waveguide devices, antenna radiation and electromagnetic scattering of arbitrarily shaped objects. Due to the finite element approach, material properties can be easily treated in a rigorous manner.

Then, in the second chapter, two domain decomposition methods have been analyzed adding their formulations to the validated FES-3D, looking forward to apply the method to large electromagnetic problems. The direct Schur complement approach has demonstrated extremely high computational accuracies, while requiring high simulation times to assemble the Schur complement matrix. The finite element tearing and interconnecting dual-primal (FETI-DP) approach, whit Robin-Robin transmission conditions have demonstrated a better decoupling between subdomains and a better iterative behavior when used as preconditioner for a Krylov subspace iterative solver. However, the Schur complement based domain decomposition allow for a better waveports truncation method, the transfinite element method, while the FETI-DP requires a the waveports not to be shared between subdomains in order to maintain subproblems decoupling. The behavior of several domain decomposition preconditioners for iterative solvers have been analyzed, and a linear complexity could be achieved for memory requirements, while the simulation times behave almost as those of direct solvers.

In the third chapter, an accurate approach for solving nonlinear microwave problems have been proposed with the first documented application of the harmonic balance finite element method to the wave equation. Several tests have been conducted, first with FES-2D then with FES-3D, to assess the spurious fields generation by nonlinear dielectrics and nonlinear ferrites. The results almost match the measurements reported in literature.

\section{Outlook}

During the years dedicated to the research illustrated within this dissertation, many open problems and relevant fields of application of the method have appeared.

First, the Schur compement based Gauss-Seidel preconditioner could be improved by a mixing of the direct approach and Krylov subspace approach. For instance, if one could replace the $\mat{A_{\Gamma\Gamma}}$ block of the preconditioner with an approximate Schur complement $\tilde{\mat{S}}$, then the number of iterations required in the GMRES($r$) solver would rapidly decrease, as it has been noticed that the substitution of $\mat{A_{\Gamma\Gamma}} \leftarrow \mat{S}$ only requires a couple of iterations to converge to numerical error. Its approximation, with the right trade-off to keep the block easily invertible, would lead to an optimal domain decomposition preconditioner, and hence the rapid solution of very large problems, while keeping the accuracy of multi-mode transfinite element method on waveports.

Also, the harmonic balance finite element method for the wave equation open the path to nonlinear problems currently solved only by non-rigorous methods. Among these there are all the passive intermodulation problems that occur at microwave frequencies. This is still one challenging problem that is still tackled without the direct solution of Maxwell's equations \cite{henrie2008prediction,lin2014design} and hence all models strictly depend on the employed device type. Also, the nonlinearities of materials at microwave frequencies can be exploited to design electronically reconfigurable devices \cite{Sigman2008, Zhang2008, Delprat2011}. Coupling of electrostatic and microwave formulations may lead to accurate modeling of these devices. At optical frequencies, nanoparticles of barium titanate oxides have demonstrated interesting capabilities of subwavelength coherent light generation when illuminated by non-coherent light \cite{pu2010nonlinear}. Accurate modeling of these nanoparticles, combined with a domain decomposition method to extend the problem dimensions may lead to very interesting photonics and biosensing applications. Same behavior is found with the emergent nanowires \cite{dutto2011nonlinear}.

These were some of the most modern challenging problems that the results of this dissertation may allow for straightforward solution, in some cases upon combining both domain decomposition schemes with nonlinear analyzes.