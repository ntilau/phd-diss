% !TEX root = ln_diss.tex
\graphicspath{{img/}}
\thispagestyle{empty}

%\AlCentroPagina{Frontespizio/UniFI_sfondo}

\begin{figure}[htbp]
\centering
\includegraphics[scale=0.5]{UniFI}
\label{fig:UniFI}
\end{figure}
\begin{center} 
\vspace{-5pt}
%\large{Department of Information Engineering\\[5pt]
%\textbf{\textsc{University of Florence}}}\\[20pt]
\textbf{\textsc{Dottorato di Ricerca in}}
\textbf{\textsc{Tecnologie Elettroniche per l'Ingegneria dell'Informazione}}
\mat\\[5pt]
in\\[5pt]
\emph{RF, MICROWAVE AND ELECTROMAGNETICS}\\[5pt]
Course XXIV\\[5pt]
\textbf{Curriculum:} Numerical methods for electromagnetics\\[5pt]



\vspace{40pt} 
\textsc{
\Large \textbf {The Finite Element Method for Computational Electromagnetics Applications}
}

\vspace{30pt}
\normalsize{
Laurent Ntibarikure\\}

\vspace{30pt} 

A Dissertation submitted in Partial Fulfillment of the Requirements for\\
the Degree of Doctor of Philosophy in the University of Florence\\
2013

\end{center}


\vspace{30pt}
\noindent \textbf{Coordinator:}\\[10pt]
\hspace*{20pt}	Prof. Gianfranco Manes\\[15pt]
\noindent \textbf{Advisors:}\\[10pt]
\hspace*{20pt}	Prof. Giuseppe Pelosi\\[10pt]
\hspace*{20pt}	Dr. Stefano Selleri



% pagina vuota dietro al frontespizio
\newpage
\thispagestyle{empty}
\hfill

\vfill
{
\footnotesize{
\noindent According to note No 1581, July 26, 2005, of the Italian Ministry of Education, 
University and Research (MIUR) and the Ministerial Decree of October 4, 2000 
Published in the \quotes{Gazzetta Ufficale} of October 24, 2000 No 249,
the present PhD thesis is classified in the scientific area ING-INF/02
}
}
\mbox{}
\newpage


%\graphicspath{{img/}}
%\thispagestyle{empty}
%
%\AddToShipoutPicture*{\AtPageCenter{\setlength\fboxsep{0pt}\makebox(0,0){\includegraphics[width=0.9\paperwidth]{LOGOUNIFI_sfondo}}}}
%
%\begin{figure}[htbp]
%\centering
%\includegraphics[width=6cm]{LOGOUNIFI}
%\label{fig:UniFI}
%\end{figure}
%\begin{center} 
%%\vspace{-5pt}
%%\large{Department of Information Engineering\\[1pt]
%%\textbf{\textsc{University of Florence}}}\\[20pt]
%\large{
%%International Doctorate Program\\[0pt]
%\textbf{\textsc{Dottorato di Ricerca in\\[5pt]}}
%\textbf{\textsc{Tecnologie Elettroniche per l'Ingegneria dell'Informazione\\[5pt]}}
%%in\\[0pt]
%Indirizzo: \quotes{\emph{RF, MICROWAVES AND ELECTROMAGNETICS}}
%\\[4pt]
%Ciclo XXVI\\[4pt]
%Coordinatore: Numerical methods for electromagnetics\\[35pt]
%}
%\textsc{\huge \textbf{Contributions~to~the~Art~of~\\\vspace{-5pt}Finite~Element~Analysis~in~\\\vspace{10pt} Electromagnetics}}\\[20pt]
%\Large{Laurent Ntibarikure}\\[25pt]
%%\normalsize
%%A Dissertation submitted in Partial Fulfillment of the Requirements for\\
%%the Degree of Doctor of Philosophy in the University of Florence\\[5pt]
%%\Large{2011-2013}
%\Large{ING-INF/02}
%\end{center}
%\vspace{30pt}
%\noindent \textbf{Coordinator:}\\[10pt]
%\hspace*{20pt}	Prof. Gianfranco Manes\\[15pt]
%\noindent \textbf{Advisors:}\\[10pt]
%\hspace*{20pt}	Prof. Giuseppe Pelosi\\[10pt]
%\hspace*{20pt}	Dr. Stefano Selleri\\[10pt]
%\begin{center}
%\Large{Anni 2011/2013}
%\end{center}
%
%%\newpage
%%\thispagestyle{empty}
%%\hfill
%%\vfill
%%{
%%\footnotesize{
%%\noindent According to note No 1581, July 26, 2005, of the Italian Ministry of Education, 
%%University and Research (MIUR) and the Ministerial Decree of October 4, 2000 
%%Published in the \quotes{Gazzetta Ufficale} of October 24, 2000 No 249,
%%the present PhD thesis is classified in the scientific area ING-INF/02
%%}
%%}
%%\mbox{}