% !TEX root = ln_diss.tex
\chapter*{Sommario}
Il metodo degli elementi finiti \`e un potente metodo per la soluzione di problemi  di valori al contorno complessi, governati da equazioni alle differenze parziali. La sua prima applicazione a problemi di ingegneria risale al 1943, anno in cui Courant divulg\`o la soluzione numerica di un problema di meccanica strutturale \cite{courant1943variational}. Da allora, vi sono stati numerosi tentativi di applicazione del metodo ad altri campi dell'ingegneria. Silvester, pioniere del metodo nel campo dell'elettromagnetismo applicato, nei suoi primi lavori datati 1969 %sulla rivista \quotes{Alta Frequenza} 
illustr\`o la possibilit\`a di risolvere problemi di guida d'onda con tale metodo \cite{silvester1969finite, coccioli1996finite}. Oggi giorno, vasto \`e il bagaglio di problemi elettromagnetici risolti e validati sperimentalmente. Molti ormai sono gli applicativi disponibili per l'analisi numerica agli elementi finiti di strutture guidanti e radianti.

Nonostante l'immenso sviluppo degli ultimi decenni, vi sono ancora molti problemi da risolvere. Nella presente tesi analizzeremo due di essi, fornendo alcune idee e tecniche risolutive. Il primo riguarda la soluzione di problemi \quotes{grandi}, irrisolubili con modesti comuni calcolatori, nel tentativo quindi di sfruttare al meglio le risorse computazionali disponibili ed oltrepassarne i limiti attuali. La strada della scomposizione di dominio \`e quindi stata impegnata. Suddividendo un  problema grande in vari sotto-problemi che per le loro dimensioni limitate risultano affrontabili singolarmente, e raccordando opportunamente le singole soluzioni \`e possible ottenere la soluzione del problema originale. Il secondo problema \`e legato all'imminente apparizione di materiali con caratteristiche elettromagnetiche intrinsecamente non lineari, ovvero dipendenti dal campo elettromagnetico in essi contenuto. L'analisi a microonde di tali mezzi con un approccio agli elementi finiti convenzionale non consente di raggiungere un sufficiente livello di accuratezza. L'analisi multiarmonica, includendo gli effetti non lineari, consente di migliorare notevolmente l'accuratezza di analisi. Un applicativo per l'analisi agli elementi finiti di problemi tridimensionali \`e stato implementato al fine di condurre le suddette ricerche.
\newpage
\thispagestyle{empty}
\cleardoublepage

\chapter*{Abstract}
The finite element method is a powerful method for the approximate solution of boundary value problems governed by partial differential equations. A really first application to structural engineering problems, dating 1943, is attributed to R. Courant \cite{courant1943variational}. Since then, there has been a lot of successful tentatives to apply the method to other fields. In particular, Silvester showed in 1969 \cite{silvester1969finite, coccioli1996finite} that waveguide modes could be easily computed with the method. His work started a long path for finite elements in electromagnetics, with multiple assessments of the method with real-world problems and gradually improving the efficiency of the algorithms. Nowadays, finite elements in computational electromagnetics has become an invaluable part in radio frequency and microwave application designs, and many packages are widely available to perform these tasks. 

However, there remain a lot of problems to be solved. In this dissertation, we have inquired in two of these. 
The first, the efficient solution of large problems which may not be solvable on a single modern computer. Domain decomposition methods have been thus investigated, these allowing to solve smaller parts of a large problem and to achieve the whole solution upon proper interconnection. Two types of domain decomposition methods have been analyzed, leading to the construction of algorithms for solving large electromagnetic problems at a nearly linear complexity. 
The other, the accurate solution of electromagnetic problems in which some materials behave nonlinearly, that is their properties vary depending on the intensity of the fields they imbue. Almost all materials behave nonlinearly and their effect is just a matter of fields intensities and accuracy requirements. In many microwave applications, the nonlinear effects, necessary for information processing and control, are still limited to lumped devices for their highly developed models. Accurate modeling of bulk or films of nonlinear materials may open the way to a new variety of controllable materials in flexible, reconfigurable, electromagnetic devices. A finite element package has been implemented to perform several tests here documented.

%Computers have also evolved during the last decades, increasing noticeably the computational power density up to the point a modern personal computer such as a low power laptop is several order more powerful than the first, room-sized, computers. Due to thermal power limitations, most of the modern processors are equipped with multiple cores, hence the algorithms are nowadays oriented to best exploit these architectures.

\vfill
\noindent \textit{Index Terms}: Electromagnetic radiation and scattering, time-harmonic fields, finite element method, domain decomposition methods, nonlinear materials, multiharmonic analysis, harmonic balance.